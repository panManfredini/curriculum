
\documentclass[11pt,a4paper,roman]{moderncv} % Font sizes: 10, 11, or 12; paper sizes: a4paper, letterpaper, a5paper, legalpaper, executivepaper or landscape; font families: sans or roman

\moderncvstyle{classic} % CV theme - options include: 'casual' (default), 'classic', 'oldstyle' and 'banking'
\moderncvcolor{green} % CV color - options include: 'blue' (default), 'orange', 'green', 'red', 'purple', 'grey' and 'black'

\usepackage{lipsum} % Used for inserting dummy 'Lorem ipsum' text into the template

\usepackage[scale=0.75]{geometry} % Reduce document margins
\setlength{\hintscolumnwidth}{2.5cm} % Uncomment to change the width of the dates column
\setlength{\makecvtitlenamewidth}{13cm} % For the 'classic' style, uncomment to adjust the width of the space allocated to your name
\nopagenumbers{}  
%----------------------------------------------------------------------------------------
%	NAME AND CONTACT INFORMATION SECTION
%----------------------------------------------------------------------------------------

\firstname{\huge Alessandro} % Your first name \\
\familyname{\huge  Manfredini } % Your last name

% All information in this block is optional, comment out any lines you don't need
\title{Curriculum Vitae}

\address{Weizmann Institute of Science \\ Dep.  Particle Physics and Astrophysics  \\234 Herzl St., Rehovot, }{Israel }
%\extrainfo{ 234 Herzl St., Rehovot}
\phone{+972-8-934-2700}
\email{alessandro.manfredini@weizmann.ac.il}
%\photo[70pt][3pt]{dummyimage.png} % The first bracket is the picture height, the second is the thickness of the frame around the picture (0pt for no frame)
%\quote{"A witty and playful quotation" - John Smith}

%----------------------------------------------------------------------------------------

\begin{document}

\makecvtitle 


%----------------------------------------------------------------------------------------
%	EDUCATION SECTION
%----------------------------------------------------------------------------------------

\section{Education}

\cvline{2014--Present}{\textbf{Postdoctoral Fellow}. \newline \emph{Weizmann Institute of Science}, Rehovot, Israel.} 
\cvline{2011--2014}{\textbf{Ph.D. Student of International Max Planck Research School.} 
		    \newline \emph{Max-Planck Institute For Physics}, Munich, Germany.  }
\cvline{2008--2010}{\textbf{Master of Science in Nuclear and Sub-nuclear Physics.} 
		   \newline \emph{University of Roma Tre}, Rome, Italy. Grade: 110/110 magna cum laude.}

\section{Postdoctoral Research:}
\cvline{}{I joined the XENON dark matter project in 2014 under the supervision of Dr.~Ran~Budnik.
	The XENON project features deep underground experiments that aim to detect dark matter with 
	a dual phase liquid xenon Time Projection Chamber. XENON1T is the most sensitive dark matter experiment to date. 
%	My main contribution in XENON was to perform data analysis with particular focus on statistical inference of data,
%	furthermore, I also contributed to the development of the XENON1T slow control system.
	}

\cvline{\textcolor{color1}{Statistics:}}{Since October 2016 I am leading and coordinating the statistical inference effort of the XENON1T experiment.
			I was part of a team that performed the statistical combination of the three science runs of the XENON100 experiment. 
			Currently one of my main interests is the development of statistical methods particularly aimed for low rate experiments.}

\cvline{\textcolor{color1}{Data Analysis:}}{ I contributed to two independent analysis campaigns being one of the main analyzers. One performing dark matter search in the 
	framework of Effective Field Theory. The other investigating inelastic dark matter scattering on $^{129}$Xe isotope. Both analyses currently passed XENON internal review, articles in preparation.}

\cvline{\textcolor{color1}{Slow Control:}}{I am part of the XENON1T slow control developers team. My main contribution was to design the safety and motion control systems 
					for a set of motors and belts used to move calibration sources in the experiment. I also developed the XENON1T PMTs high voltage module
					controller, and the safety system of the experiment's water recirculation facility.}


\section{Research during Ph.D.:}
\cvline{}{I joined the ATLAS experiment in 2011 under the supervision of  Prof. Siegfried Bethke and Dr. Sandra Kortner.  
	ATLAS is a general purpose particles detector and part of the Large Hadron Collider project.}

\cvline{\textcolor{color1}{Thesis Title:}} {"Search for Neutral MSSM Higgs Bosons in  $A/h/H \rightarrow \tau^+\tau^- \rightarrow e\mu + 4\nu$ Decays with the ATLAS Detector." }

\cvline{\textcolor{color1}{Data Analysis}}{I was part of the ATLAS Beyond Standard Model Higgs sub-group. 
			I contributed to the neutral MSSM Higgs boson search of	which I was one of the main analyzer. 
			This analysis  considered the Higgs decaying into taus and their subsequent decay in electron 
			or muon. My contribution was focused on: the estimation of the backgrounds and their systematics, the  implementation of 
			the probability	model for hypothesis testing, and on studies aimed to improve the signal sensitivity at
			low mass employing flavor tagging techniques.
			}

\cvline{ \textcolor{color1}{Teaching}}{
			I supervised a laboratory experience named "Trace measurements of cosmic ray
			muons with drift tube chambers" for the Technical University of Munich.}



\nocite{*}
\bibliographystyle{unsrt}
%\bibliographystyle{plain}
\bibliography{cv_manfredini}

\section{Articles in Preparation:}
\cvline{} {\textcolor{color2}{XENON Collaboration. Search for WIMP Inelastic Scattering Off Xenon Nuclei With XENON100. \emph{ Has passed internal review.}}}
\cvline{} {\textcolor{color2}{XENON Collaboration. Effective Field Theory Approach to Scattering of Dark Matter in XENON100 Detector 225 live days run. \emph{ Has passed internal review.}}}
\section{Contribution to Conference and Workshops:}

\cvline{Nov. 2016} {\emph{ International Workshop on "Double Beta Decay and Underground Science"}. (Osaka, Japan). 
		Talk on behalf of XENON collaboration: "Status of the XENON1T Experiment".}

\cvline{May 2013}{\emph{ HSG6 - ATLAS Higgs beyond the Standard Model searches Workshop.} (Tel Aviv,Israel).  Talk title: "Highlights and Future Prospects for the leplep channel".}

\cvline{Aug. 2013}{\emph{ 21 st International Conference on Supersymmetry and Unification of Fundamental Interactions.} 
		Talk on behalf of the ATLAS collaboration, title: "Beyond-the-Standard Model Higgs Physics using the ATLAS detector".}
\cvline{Jan. 2012} {\emph{ATLAS - Flavor Tagging Workshop.} (Nijmegen, Netherlands). \newline Talk title: "Low pT b-tagging with track jets".}
%\cvline{Mar. 2012 } {Deutsche Physikalische Gesellschaft, Conference. (Gottingen, Germany).Talk title: "b-tagging performance optimization for neutral MSSM Higgs bosons search with the ATLAS detector".}
%\cvline{Nov. 2011 }{ Hadron Collider Physics Symposium 2011. (Paris, France).Poster title: "Study of muon isolation."}

%\section{Languages}

%\cvitemwithcomment{English}{Mothertongue}{}

\end{document}

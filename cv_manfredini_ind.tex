
\documentclass[11pt,a4paper,roman]{moderncv} % Font sizes: 10, 11, or 12; paper sizes: a4paper, letterpaper, a5paper, legalpaper, executivepaper or landscape; font families: sans or roman

\moderncvstyle{classic} % CV theme - options include: 'casual' (default), 'classic', 'oldstyle' and 'banking'
\moderncvcolor{green} % CV color - options include: 'blue' (default), 'orange', 'green', 'red', 'purple', 'grey' and 'black'

\usepackage{lipsum} % Used for inserting dummy 'Lorem ipsum' text into the template

%\usepackage[]{hyperref}
%\usepackage{soul}
\usepackage[normalem]{ulem}
\newcommand\bul{\bgroup\markoverwith{\textcolor{blue}{\rule[-0.5ex]{2pt}{0.4pt}}}\ULon}

\usepackage[scale=0.75]{geometry} % Reduce document margins
\setlength{\hintscolumnwidth}{2.5cm} % Uncomment to change the width of the dates column
\setlength{\makecvtitlenamewidth}{13cm} % For the 'classic' style, uncomment to adjust the width of the space allocated to your name
\nopagenumbers{}  
%----------------------------------------------------------------------------------------
%	NAME AND CONTACT INFORMATION SECTION
%----------------------------------------------------------------------------------------

\firstname{\huge Alessandro} % Your first name \\
\familyname{\huge  Manfredini } % Your last name

% All information in this block is optional, comment out any lines you don't need
\title{Curriculum Vitae}

\address{University Of Zurich \\ Department of Physics \\ Winterthurerstrasse 190, \\ Z{\"u}rich, }{Switzerland }
%\extrainfo{ 234 Herzl St., Rehovot}
\phone{+41 783 285 463}
\email{a.manfredini.work@gmail.com}
\extrainfo{Skype: manfredini\_alessandro} % \\ Git: panManfredini}
\photo[62pt][3pt]{ale_cropped_color.jpg} % The first bracket is the picture height, the second is the thickness of the frame around the picture (0pt for no frame)
%\quote{"A witty and playful quotation" - John Smith}

%----------------------------------------------------------------------------------------

\begin{document}

\makecvtitle 

\section{Personal Details}
\cvline{\textbf{Birth}}{\emph{November 11th 1985}, Rome, Italy.} 
\cvline{\textbf{Nationality}}{\emph{Italian.}} 

%----------------------------------------------------------------------------------------
%	EDUCATION SECTION
%----------------------------------------------------------------------------------------

\section{Education}

\cvline{2018--Present}{\textbf{Postdoctoral Fellow in Astroparticle Physics}. \newline \emph{University Of Zurich}, Zurich, Switzerland.} 
\cvline{2014--2018}{\textbf{Postdoctoral Fellow  in Astroparticle Physics}. \newline \emph{Weizmann Institute of Science}, Rehovot, Israel.} 
\cvline{2011--2014}{\textbf{Ph.D. Student of International Max Planck Research School.} 
		    \newline \emph{Max-Planck Institute For Physics}, Munich, Germany.  }
\cvline{2008--2010}{\textbf{Master of Science in Nuclear and Sub-nuclear Physics.} 
		   \newline \emph{University of Roma Tre}, Rome, Italy. Grade: 110/110 magna cum laude.}

\section{Coordination and Awards}
\cvline{2016--2018}{ I was the XENON1T experiment's \textbf{statistical inference coordinator}.} 
\cvline{2019 --}{ DARWIN experiment's "electrodes and high voltage" sub-group coordinator.} 
\cvline{2017}{Awarded \emph{"senior postdoctoral fellow"} at Weizmann Institute of Science.} 

\section{Skills}
\cvline{}{My main expertise are in \textbf{statistics} and \textbf{data intensive analysis}. I am also experienced with automation software
(PLC and SCADA systems programming). Proficient in C++, C\#, Javascript and Python. Good knowledge of SQL and NoSQL databases, 
data streaming platform like Kafka as well as micro-service orchestration with Kubernetes.}



\section{Second Postdoc}
\cvline{\textcolor{color1}{2018--Present}} { I joined the group of Prof.~Laura~Baudis as co-leader of an R\&D project for the future DARWIN dark matter detector
(\bul{\href{https://darwin.physik.uzh.ch/}{more info here}}). I am responsible for the design of a fully automated experiment monitoring and control system
based on industrial Programmable Logic Controllers (PLC) and state of the art of big-data tools. I designed innovative IoT gateways (\bul{\href{https://opc-proxy.readthedocs.io/en/latest/intro.html}{OPC-Proxy}} and \bul{\href{https://github.com/panManfredini/IOTpy}{IoTpy}}) that allows to connect  the data-taking sensors with the micro-services infrastructure of our back-end. 
We use Kubernetes to orchestrate a series of services that includes Prometheus, Grafana, NodeJs and Apache Kafka. Finally, I designed the graphical user interface of our system,
allowing to control the experiment remotely from a browser.}

\section{First Postdoc}
\cvline{\textcolor{color1}{2014--2018}}{I joined the \bul{\href{https://en.wikipedia.org/wiki/XENON}{XENON}} dark matter project in the group of Prof.~Ran~Budnik.
%	The \bul{\href{https://en.wikipedia.org/wiki/XENON}{XENON project}} features deep underground experiments that aim to detect dark matter with a
%	liquid xenon Time Projection Chamber. 
	The XENON collaboration consist of more than 100 physicist from all over the world. 
%	My main contribution in XENON was to perform data analysis with particular focus on statistical inference of data,
%	furthermore, I also contributed to the development of the XENON1T slow control system.
	}

\cvline{\textcolor{color1}{Statistics:}}{For two years I was co-leading the statistical inference team of the XENON1T experiment, where I performed 
hypothesis testing with computation of the confidence intervals and developed the statistical model of the experiment.}

\cvline{\textcolor{color1}{Data Analysis:}}{ I contributed with two independent Analyses. One performing dark matter search in the 
	framework of Effective Field Theory. The other investigating inelastic dark matter scattering on $^{129}$Xe isotope.}

\cvline{\textcolor{color1}{Slow Control:}}{I was part of the XENON1T slow control developers team. My main contribution was to design the safety and motion control systems 
					for a set of motors and belts used to move calibration sources in the experiment. I also developed the high voltage module
					controller, and the safety system of the experiment's water recirculation facility.}


\section{Research during Ph.D.}
\cvline{}{I joined the \bul{\href{https://atlas.cern/}{ATLAS}} experiment at CERN in 2011 under the supervision of Dr.~Sandra Kortner.  
%	\bul{\href{https://atlas.cern/}{ATLAS}} is a general purpose particles detector and part of the Large Hadron Collider project. 
	The ATLAS collaboration consist of thousands of physicist all over the world.}

\cvline{\textcolor{color1}{Thesis Title:}} {"Search for Neutral MSSM Higgs Bosons in  $A/h/H \rightarrow \tau^+\tau^- \rightarrow e\mu + 4\nu$ Decays with the ATLAS Detector." }

\cvline{\textcolor{color1}{Data Analysis}}{I was part of the ATLAS Beyond Standard Model Higgs sub-group. 
			I contributed to the neutral MSSM Higgs boson search of	which I was one of the main analyzer. 
			%This analysis  considered the Higgs decaying into taus and their subsequent decay in electron 
			%or muon. 
			My contribution was focused on the estimation of the backgrounds and their systematics, the  implementation of 
			the probability	model for hypothesis testing, and on studies aimed to improve the signal sensitivity at
			low mass employing flavor tagging techniques.}



\section{Software Skills}
\cvline{\textcolor{color1}{Data Analysis}}{ Proficient in C++, Deep knowledge of ROOT framework, ROOSTAT and RooFit. Good knowledge of Python and C\#.}
\cvline{\textcolor{color1}{Automation}}{ PLC ladder diagram, structured text and SCADA programming.}
\cvline{\textcolor{color1}{Web}}{ Proficient in HTML, CSS, JavaScript, SQL and NoSQL database. Server-side (NodeJS/Django) and client-side programming. 
Good knowledge of Kubernetes and the Kafka streaming platform.}

\section{My Git Repositories}
\cvline{}{
	\begin{tabular}{p{3cm} p{3mm} l}
		\bul{\href{https://github.com/XENON1T/Xephyr}{Xephyr}} & - & A statistical framework (C++). \\ 
		\bul{\href{https://github.com/WebComponentHelpers/ImperaJS}{ImperaJS}} & - &  An app-state managment framework (Javascript). \\
		\bul{\href{https://github.com/WebComponentHelpers/Brick}{Brick-Element}} & - &  A web-component generator (Javascript). \\
		\bul{\href{https://github.com/opc-proxy}{OPC-Proxy}} & - &  A modular OPC gateway (C\#). \\
		\bul{\href{https://github.com/JaS-HMI/jashmi}{JaS-HMI}} & - & A Javascript Human-Machine-Interface framework. \\
		\bul{\href{https://github.com/panManfredini/IOTpy}{IOTpy}} & - & A Python framework to  expose devices trough REST API. \\
	\end{tabular}
	}

%\cvline{\textcolor{color1}{Statistics}}{ A statistical framework: \bul{\href{https://github.com/XENON1T/Xephyr}{Xephyr}}.}
%\cvline{\textcolor{color1}{Web}}{An app-state managment framework: \bul{\href{https://github.com/WebComponentHelpers/ImperaJS}{ImperaJS}}. \newline
%A custom-element generator: \bul{\href{https://github.com/WebComponentHelpers/Brick}{Brick-Element}}.}
%\cvline{\textcolor{color1}{IoT}}{A Javascript HMI framework:  \bul{\href{https://github.com/JaS-HMI/jashmi}{JaS-HMI}}. \newline
%A modular OPC gateway:~\bul{\href{https://github.com/opc-proxy}{OPC-Proxy}}.}

\section{Language Skills}
\cvline{\textcolor{color1}{Mother tongue}}{ Italian}

\cvline{\textcolor{color1}{English}}{ Fluent - (CEFR C1) } 
\cvline{\textcolor{color1}{Spanish}}{ Good understanding - (CEFR B1)}
\cvline{\textcolor{color1}{German}}{Basics - (CEFR A2)}
\end{document}

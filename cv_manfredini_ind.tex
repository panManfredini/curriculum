
\documentclass[11pt,a4paper,roman]{moderncv} % Font sizes: 10, 11, or 12; paper sizes: a4paper, letterpaper, a5paper, legalpaper, executivepaper or landscape; font families: sans or roman

\moderncvstyle{classic} % CV theme - options include: 'casual' (default), 'classic', 'oldstyle' and 'banking'
\moderncvcolor{green} % CV color - options include: 'blue' (default), 'orange', 'green', 'red', 'purple', 'grey' and 'black'

\usepackage{lipsum} % Used for inserting dummy 'Lorem ipsum' text into the template

%\usepackage[]{hyperref}
%\usepackage{soul}
\usepackage[normalem]{ulem}
\newcommand\bul{\bgroup\markoverwith{\textcolor{blue}{\rule[-0.5ex]{2pt}{0.4pt}}}\ULon}

\usepackage[scale=0.75]{geometry} % Reduce document margins
\setlength{\hintscolumnwidth}{2.5cm} % Uncomment to change the width of the dates column
\setlength{\makecvtitlenamewidth}{13cm} % For the 'classic' style, uncomment to adjust the width of the space allocated to your name
\nopagenumbers{}  
%----------------------------------------------------------------------------------------
%	NAME AND CONTACT INFORMATION SECTION
%----------------------------------------------------------------------------------------

\firstname{\huge Alessandro} % Your first name \\
\familyname{\huge  Manfredini } % Your last name

% All information in this block is optional, comment out any lines you don't need
\title{Curriculum Vitae}

%\address{Limmattalstrasse 316, \\ Z{\"u}rich, }{Switzerland }
%\extrainfo{ 234 Herzl St., Rehovot}
%\phone{+41 783 285 463}
%\email{ale.manfredini@proton.me}
%\extrainfo{Skype: manfredini\_alessandro} % \\ Git: panManfredini}
\photo[62pt][3pt]{ale_cropped_color.jpg} % The first bracket is the picture height, the second is the thickness of the frame around the picture (0pt for no frame)
%\quote{"A witty and playful quotation" - John Smith}

%----------------------------------------------------------------------------------------

\begin{document}

\makecvtitle 

\section{Personal Details}
\cvline{\textbf{Birth}}{November 11th 1985, Rome, Italy.} 
\cvline{\textbf{Nationality}}{Italian.}
\cvline{\textbf{Address}}{Limmattalstrasse 316, 8049 Z{\"u}rich, Switzerland. }
\cvline{\textbf{E-mail}}{ale.manfredini@proton.me}
\cvline{\textbf{Phone}}{+41 783 285 463}

%----------------------------------------------------------------------------------------
%	EDUCATION SECTION
%----------------------------------------------------------------------------------------

\section{Work Experience}

\cvline{2021--Present}{\textbf{Software Engineer}. \newline \emph{Arktis Radiation Detectors}, Z{\"u}rich, Switzerland.} 
\cvline{2018--2021}{\textbf{Postdoctoral Fellow in Experimental Physics}. \newline \emph{University Of Z{\"u}rich}, Z{\"u}rich, Switzerland.} 
\cvline{2014--2018}{\textbf{Postdoctoral Fellow  in Experimental Physics}. \newline \emph{Weizmann Institute of Science}, Rehovot, Israel.} 

\section{Education}
\cvline{2011--2014}{\textbf{Ph.D. Student in Experimental Particle Physics.} 
		    \newline \emph{Max-Planck Institute For Physics}, Munich, Germany.  }
\cvline{2008--2010}{\textbf{Master of Science in Nuclear and Sub-nuclear Physics.} 
		   \newline \emph{University of Roma Tre}, Rome, Italy.}

%\section{Coordination and Awards}
%\cvline{2016--2018}{ I was the XENON1T experiment's statistical inference coordinator.} 
%\cvline{2019 --}{ DARWIN experiment's "electrodes and high voltage" sub-group coordinator.} 
%\cvline{2017}{Awarded \emph{"senior postdoctoral fellow"} at Weizmann Institute of Science.} 

% Two years experience of software development for enviromental monitors in the field of radiation protection.
% Designed software to interface with mass flowmeters, valves etc.

\section{About Me}
\cvline{}{
My main expertise is in data intensive statistical analyses and the Internet of
Things. For the last two years I was employed in the development/maintenance
of the software stack for environmental monitors in the field of radiation protection.
I have been professionally programming in C++, Python and JavaScript for more than six years. 
I am experienced with micro-service architecture design, Docker and orchestration with Kubernetes. 
I have experience with SQL and NoSQL databases and modern streaming libraries like Apache Arrow.
I'm passionate about code testing, and once I've finished programming, I love to go rock climbing. }

\section{Software Engineer at Arktis}
\cvline{\textcolor{color1}{2021--Present}} {
\bul{\href{https://www.arktis-detectors.com}{Arktis}} is a company that designs innovative solutions for radiation detection.
The most common products are Radiation Portal Monitors, which are systems equipped with a variety of sensors (radiation detectors, lidar, camera, temperature, humidity) 
and an embedded PC for realtime data analysis.}
\cvline{\textcolor{color1}{Software Stack:}}{Being a small company I've been developing and maintaining the entire software stack, 
from sensor readout and detector calibration to event data analysis, web interface and user experience. 
\emph{
My main responsibility is to design algorithms that efficiently identify in real-time the radioactive isotopes based on the recorded energy spectra}.}
\cvline{\textcolor{color1}{Machine Learning:}}{ I've been designing and training deep learning models (Keras/Tensorflow)
for spectral identification using radiation detector data. I've also developed a neural-network that classifies road traffic conditions (in the vicinity of the sensors)
using camera images.}

\section{Research During Second Postdoc}
\cvline{\textcolor{color1}{2018--2021}} { I joined the group of Prof.~Laura~Baudis as co-leader of an ERC funded R\&D project for the future DARWIN dark matter detector
(\bul{\href{https://darwin.physik.uzh.ch/}{more info here}}). I was responsible for the design of a fully automated experiment monitoring and control system
based on industrial Programmable Logic Controllers (PLC) and state of the art of big-data tools. I designed innovative IoT gateways (\bul{\href{https://opc-proxy.readthedocs.io/en/latest/intro.html}{OPC-Proxy}} and \bul{\href{https://github.com/panManfredini/IOTpy}{IoTpy}}) that allows to connect  the data-taking sensors with the micro-services infrastructure of our back-end. 
We use Kubernetes to orchestrate a series of services that includes Prometheus, Grafana, NodeJs and Apache Kafka. Finally, I designed the graphical user interface of our system,
allowing to control the experiment remotely from a browser.}

\section{Research During First Postdoc}
\cvline{\textcolor{color1}{2014--2018}}{I joined the \bul{\href{https://en.wikipedia.org/wiki/XENON}{XENON}} dark matter project in the group of Prof.~Ran~Budnik.
%	The \bul{\href{https://en.wikipedia.org/wiki/XENON}{XENON project}} features deep underground experiments that aim to detect dark matter with a
%	liquid xenon Time Projection Chamber. 
	The XENON collaboration consist of more than 100 physicist from all over the world. 
%	My main contribution in XENON was to perform data analysis with particular focus on statistical inference of data,
%	furthermore, I also contributed to the development of the XENON1T slow control system.
	}

\cvline{\textcolor{color1}{Statistics:}}{For two years I was co-leading the statistical inference team of the XENON1T experiment, where I performed 
hypothesis testing with computation of the confidence intervals and developed the statistical model of the experiment.}

%\cvline{\textcolor{color1}{Data Analysis:}}{ I contributed with two independent Analyses. One performing dark matter search in the 
%framework of Effective Field Theory. The other investigating inelastic dark matter scattering on $^{129}$Xe isotope.}

\cvline{\textcolor{color1}{Slow Control:}}{I was part of the XENON1T slow control developers team. My main contribution was to design the safety and motion control systems 
					for a set of motors and belts used to move calibration sources in the experiment. I also developed the high voltage module
					controller, and the safety system of the experiment's water recirculation facility.}


\section{Research During Ph.D.}
\cvline{}{I joined the \bul{\href{https://atlas.cern/}{ATLAS}} experiment at CERN in 2011 under the supervision of Dr.~Sandra Kortner.  
%	\bul{\href{https://atlas.cern/}{ATLAS}} is a general purpose particles detector and part of the Large Hadron Collider project.
I was part of the ATLAS Beyond Standard Model Higgs sub-group. I contributed to the neutral MSSM Higgs boson search of	which I was one of the main analyzer. 
My contribution was focused on the estimation of the backgrounds and their systematics, the  implementation of the probability	model for hypothesis testing, 
and on studies aimed to improve the signal sensitivity at low mass employing flavor tagging techniques.}

\cvline{\textcolor{color1}{Thesis Title:}} {"Search for Neutral MSSM Higgs Bosons in  $A/h/H \rightarrow \tau^+\tau^- \rightarrow e\mu + 4\nu$ Decays with the ATLAS Detector." }



\section{Software Skills}
\cvline{\textcolor{color1}{Programming Languages}}{Proficient in C++, Python, JavaScript, TypeScript, HTML, CSS.  Experience with C\#.}
\cvline{\textcolor{color1}{Data Analysis and Statistics}}{ Proficient with SciPy, NumPy and Pandas libraries.}
\cvline{\textcolor{color1}{Deep Learning}}{ Basic knowledge of Keras and TensorFlow.}
\cvline{\textcolor{color1}{Web Tools}}{SQL (PostgreSQL) and NoSQL (MongoDB) databases. Server-side programming with NodeJS, Twisted and Django. Experience with Websocket and SSE. Experience with the  lit-element front-end framework (for custom-elements) and Redux. }
\cvline{\textcolor{color1}{Communication Protocols}}{RS232, TCP-IP sockets, gRPC, Arrow, OPC, MQTT.}
\cvline{\textcolor{color1}{Monitoring}}{Experience with Prometheus, AlertManager, InfluxDB and Grafana.}
\cvline{\textcolor{color1}{Cloud}}{Experience with Google Cloud Platform, Kubernetes, Docker.} 
\cvline{\textcolor{color1}{Big Data}}{Operative knowledge of Apache Kafka.} 
\cvline{\textcolor{color1}{Automation}}{ Experience with PLC ladder diagram and structured text programming.}

\section{My Git Repositories}
\cvline{}{
	\begin{tabular}{p{3cm} p{3mm} l}
		\bul{\href{https://github.com/XENON1T/Xephyr}{Xephyr}} & - & A statistical framework (C++). \\ 
		\bul{\href{https://github.com/WebComponentHelpers/ImperaJS}{ImperaJS}} & - &  An app-state managment framework (Javascript). \\
		\bul{\href{https://github.com/WebComponentHelpers/Brick}{Brick-Element}} & - &  A web-component generator (Javascript). \\
		\bul{\href{https://github.com/opc-proxy}{OPC-Proxy}} & - &  A modular OPC gateway (C\#). \\
		\bul{\href{https://github.com/JaS-HMI/jashmi}{JaS-HMI}} & - & A Javascript Human-Machine-Interface framework. \\
		\bul{\href{https://github.com/panManfredini/IOTpy}{IOTpy}} & - & A Python framework to  expose devices trough REST API. \\
	\end{tabular}
	}

%\cvline{\textcolor{color1}{Statistics}}{ A statistical framework: \bul{\href{https://github.com/XENON1T/Xephyr}{Xephyr}}.}
%\cvline{\textcolor{color1}{Web}}{An app-state managment framework: \bul{\href{https://github.com/WebComponentHelpers/ImperaJS}{ImperaJS}}. \newline
%A custom-element generator: \bul{\href{https://github.com/WebComponentHelpers/Brick}{Brick-Element}}.}
%\cvline{\textcolor{color1}{IoT}}{A Javascript HMI framework:  \bul{\href{https://github.com/JaS-HMI/jashmi}{JaS-HMI}}. \newline
%A modular OPC gateway:~\bul{\href{https://github.com/opc-proxy}{OPC-Proxy}}.}

\section{Language Skills}
\cvline{\textcolor{color1}{Mother tongue}}{ Italian}

\cvline{\textcolor{color1}{English}}{ Fluent - (CEFR C1) } 
\cvline{\textcolor{color1}{German}}{ Basic - (CEFR A2/B1)}
\end{document}
